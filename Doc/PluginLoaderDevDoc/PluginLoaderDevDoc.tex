% !TEX TS-program = pdflatex
% !TEX encoding = UTF-8 Unicode

% This is a simple template for a LaTeX document using the "article" class.
% See "book", "report", "letter" for other types of document.

\documentclass[11pt]{report} % use larger type; default would be 10pt

\usepackage[utf8]{inputenc} % set input encoding (not needed with XeLaTeX)

%%% Examples of Article customizations
% These packages are optional, depending whether you want the features they provide.
% See the LaTeX Companion or other references for full information.

%%% PAGE DIMENSIONS
\usepackage{geometry} % to change the page dimensions
\geometry{a4paper} % or letterpaper (US) or a5paper or....
% \geometry{margin=2in} % for example, change the margins to 2 inches all round
% \geometry{landscape} % set up the page for landscape
%   read geometry.pdf for detailed page layout information

\usepackage{graphicx} % support the \includegraphics command and options

% \usepackage[parfill]{parskip} % Activate to begin paragraphs with an empty line rather than an indent

%%% PACKAGES
\usepackage{booktabs} % for much better looking tables
\usepackage{array} % for better arrays (eg matrices) in maths
\usepackage{paralist} % very flexible & customisable lists (eg. enumerate/itemize, etc.)
\usepackage{verbatim} % adds environment for commenting out blocks of text & for better verbatim
\usepackage{subfig} % make it possible to include more than one captioned figure/table in a single float

%%%Personal packages
\usepackage{moreverb}
\usepackage{latexsym}

% These packages are all incorporated in the memoir class to one degree or another...

%%% HEADERS & FOOTERS
\usepackage{fancyhdr} % This should be set AFTER setting up the page geometry




\pagestyle{fancy} % options: empty , plain , fancy
\renewcommand{\headrulewidth}{0pt} % customise the layout...
\lhead{}\chead{}\rhead{}
\lfoot{}\cfoot{\thepage}\rfoot{}

%%% SECTION TITLE APPEARANCE
\usepackage{sectsty}
\allsectionsfont{\sffamily\mdseries\upshape} % (See the fntguide.pdf for font help)
% (This matches ConTeXt defaults)

%%% ToC (table of contents) APPEARANCE
\usepackage[nottoc,notlof,notlot]{tocbibind} % Put the bibliography in the ToC
\usepackage[titles,subfigure]{tocloft} % Alter the style of the Table of Contents
\renewcommand{\cftsecfont}{\rmfamily\mdseries\upshape}
\renewcommand{\cftsecpagefont}{\rmfamily\mdseries\upshape} % No bold!


%%% END Article customizations

%%% The "real" document content comes below...

\title{Developer documentation}
\author{\textsc{Wollenburger} Antoine}
%\date{} % Activate to display a given date or no date (if empty),
         % otherwise the current date is printed 

\begin{document}
\maketitle

\chapter{Create your first plugin}

The first thing you need is a new project. Name it at your conveniance.  For the example, we name it ``HelloWorld''.

\section{Fix the Java Build Path}

Secondly add ``PluginLoader.jar'' to the Java Build Path (right clic on th project, Properties, Java Build Path).
\begin{figure}[h]
	\caption{\label{buildpath} Fixing Java Build Path}
	\centering
	\includegraphics[width=0.8\textwidth]{images/BuildPath.PNG}
\end{figure}

\section{Saying ``Hello World''}

Before creating a new class, you must create a new package. In this example we take ``com.ornicare.helloworld''. Then create a new class and make it an extend of PluginRunnable and create a run method : it's the hook for the platform. It must look like this : \begin{verbatimtab}package com.ornicare.helloworld;

import com.space.plugin.PluginRunnable;

public class MainClass extends PluginRunnable {
	
	@Override
	public void run() {
		System.out.println("Hello world !");
	}
}\end{verbatimtab}

\section{Generate the propertie file}

Create a new file named ``plugin.properties'' into the project. In this file add : 
\begin{itemize}
	\item The plugin name : ``name = HelloWorld"
	\item The hook's class path : ``main = com.ornicare.helloworld.MainClass"
	\item The runnable attribut : ``runnable = true"
\end{itemize}

\section{Exporting your first plugin}

To run this, you need to create a jar of your plugin : ``File \textgreater{} Export \textgreater{} Java/JAR File". Put the created jar into the plugin directory and run the plugin loader.

\chapter{Intricated plugins}

\section{Using another plugin}

For this example, we are going to use the ``CConsole'' plugin :  it's just a plugin to display the java console (and few other things). So put ``CConsole.jar" into the plugin directory. Add it to your project Java Build Path. And use it. For example : \begin{verbatimtab}package com.ornicare.helloworld;

import cconsole.CConsole;

import com.space.plugin.PluginRunnable;

public class MainClass extends PluginRunnable {
	
	@Override
	public void run() {
		CConsole.load();
		CConsole.println("Hello world !");
	}
}\end{verbatimtab}
Then you need to add the following line to your properties file : ``depend = CConsole".

\chapter{Make a regular ``plugin.properties"}

You are allowed to use the following attributes : 
\begin{itemize}
	\item The plugin name : ``name = your\_plugin\_name"
	\item The hook's class path : ``main = package.hook\_class\_name"
	\item The runnable attribut : ``runnable = true$\backslash$false"
	\item If you want a single instance : ``singleton = true$\backslash$false"
	\item Make your plugin lazy : ``lazy = true$\backslash$false"
	\item Indicate dependencies to others plugins : ``depend = plugin1, plugin2, ..."
	\item Launch automatically : ``launch = true$\backslash$false". The plugin need to be runnable.
\end{itemize}

\chapter{Advanced functions}

\section{Advanced functions for all plugins}

\subsection{getPlugin}
\subsection{getPluginList}

\section{Advanced functions for a runnable plugin}

\section{Advanced functions for a content provide plugin}

\subsection{getObject}


\end{document}
